\section{Introducing real-time in SystemJ}
\label{sec:intr-real-time}

We introduce a single \textit{derived} statement called
\texttt{wait\-\_inbetween (M..N)}, built from kernel constructs, in the
SystemJ language for real-time control. The other two constructs
\texttt{wait\-\_atleast (M)} and \mbox{\texttt{wait\_exact (M)}} are
variants of this single derived construct.

% The resolution of the delay statement is of secondary concern and is
% dependent upon the execution platform.


\subsection{Semantics of the \texttt{wait\_inbetween} statement}
\label{sec:semant-delay-stat}

% \begin{figure}[t!]
%   \centering
%   \includegraphics[scale=0.6]{./sem}
%   \caption{Pictorial representation of the semantics of the delay
%     statement and its variants}
%   \label{fig:5}
% \end{figure}

Given a  SystemJ program:  \texttt{wait\_inbetween (M..N); p},  where $M
\in  \mathbb{Q}^{>0}$  and $N  \in  \mathbb{Q}^{>0}$,  statement $p$  is
executed after  real-time postponement  of $\tau$  units, such  that, $M
\leq \tau \leq N$.

We also introduce two variants of the derived statement
\texttt{wait\_inbetween}.
\begin{enumerate*}
\item Given a SystemJ program \texttt{wait\_atleast (M); p}, where $M
  \in \mathbb{Q}^{>0}$, statement $p$ is executed after real-time
  postponement of $\tau$ units, such that, $M \leq \tau < \infty$. It is
  important to note that the lower bound of the wait construct; $M$ is
  \textit{not} an exact postponement, but rather the control is allowed
  to proceed to the next statement anytime after waiting for \textit{at
    least} $M$ times units.
\item Given a SystemJ program \texttt{wait\_exact (M); p}, where $M \in
  \mathbb{Q}^{>0}$, statement $p$ is executed after real-time
  postponement of $\tau$ units, such that, $M \leq \tau \leq M$. In this
  variant, the waiting time is exact.
\end{enumerate*}

% Figure~\ref{fig:5} pictorially represents the semantics of these delay
% constructs. The construct \texttt{delay (M..N)} requires that upon
% execution, the control-flow is suspended for \textit{at least} $M$ units
% and \textit{at most} $N$ units, depicted as bounded box in
% Figure~\ref{fig:5}. There is no upper bound for the delay variant
% \texttt{delay(M)}, which requires that the control-flow suspends for
% \textit{at least} $M$ units, but the following statement can be executed
% after some countable delay, depicted with the unbounded box in
% Figure~\ref{fig:5}. Finally, \texttt{delay (M..M)} requires that there
% is an \textit{exact} delay of $M$ units, i.e., the control-flow suspends
% execution for \textit{at least} and \textit{at most} $M$ units.

\subsection{Rewriting the \texttt{wait\_inbetween} statement}
\label{sec:rewr-delay-stat}

The introduced \texttt{wait\_inbetween} construct is not a kernel
construct, but a \textit{derived} construct built from the kernel
constructs (Table~\ref{tab:1}) in SystemJ. Figure~\ref{fig:3} gives the
rewrite of the \texttt{wait\_inbetween} construct to kernel statements.

\begin{figure}[tb]
	\vspace{-10pt}
    \begin{minipage}{\textwidth}
      \begin{lstlisting}[style=sysj,morekeywords={abort,await,emit,present,trap,pause,exit,delay,suspend}]
trap(T){
 int x = 0;
 while(true){
  x = x + 1;
  pause;
  if(x == d) exit (T); //wait for "d" ticks
 } // d is the number of logical ticks calculated
} // by Algorithm 1
\end{lstlisting}
    \end{minipage}
    \caption{The rewrite of \texttt{wait\_inbetween} construct}
    \label{fig:3}
\end{figure}

The fundamental observation is that real-time is converted into logical
time via the \texttt{pause} construct. The rewrite basically maps
real-time back to the elegant logical notion of time. The rewrite
\textit{waits} a certain number of logical ticks, before proceeding to
the next statement. The number of logical ticks \texttt{d} to
\textit{wait} (Figure~\ref{fig:3}) is determined by the compiler
statically at compile time. The value of \texttt{d} is intricately tied
with the WCRT and BCRT of the program and hence the execution platform.

\subsection{Finding the logical wait \texttt{d}}
\label{sec:find-logic-delay}

\begin{algorithm}[t!]
  \begin{minipage}{1.0\linewidth}
    \SetAlgoLined
    \KwData{WCRT, BCRT, $M \in \mathbb{Q}^{>0}$, $N \in \mathbb{Q}^{>0}$}
    \KwResult{d}
    let $l_1 \leftarrow \lceil \frac{M}{WCRT} \rceil$\;
    let $l_2 \leftarrow \lceil \frac{M}{BCRT} \rceil$\;
    let $u_1 \leftarrow \lfloor \frac{N}{WCRT} \rfloor$\;
    let $u_2 \leftarrow \lfloor \frac{N}{BCRT} \rfloor$\;
    let $F:(l_1,u_1) \rightarrow S_1$\;
    let $F:(l_2,u_2) \rightarrow S_2$\;
    let $D \leftarrow S_1 \cap S_2$\;
    \Return (some $d \in D$)\;
    \caption{Finding the value of \texttt{d}}
    \label{alg:1}
  \end{minipage}
\end{algorithm}

The computation of \texttt{d} is shown in Algorithm~\ref{alg:1}. This
algorithm is carried out for each CD (in SystemJ) or a synchronous
program individually. \textit{The fundamental observation is -- the
  reaction time for each logical tick is elastic -- varying only between
  the BCRT and the WCRT, thus any number of logical ticks \texttt{d}
  that map to the required real-time \mbox{\texttt{wait\_inbetween
      (M..N)}} should be chosen in such a way that they are invariant to
  this elasticity.}

Our algorithm takes as input: WCRT, BCRT, and the lower and upper bounds
$M$ and $N$, respectively, of the \texttt{wait\_inbetween} construct. We
first divide $M$ and $N$ with the BCRT and WCRT, respectively. This
division gives us the number of individual logical ticks, at the fastest
(BCRT) and slowest (WCRT) program execution speeds, required to postpone
the CD (or synchronous program) by the real-time specification. We
always $ceil$ when dividing $M$ and $floor$ when dividing $N$ to make
sure that the resultant values are integers (in domain
$\mathbb{N}^{>0}$) and these functions guarantee that the resultant
logical ticks result in real-time postponements between the required
range $[M,N]$. Next, a function $F$ maps these calculated values to a
set of equidistant integer points (values) separated by a unit value --
these points represent all the logical ticks running at the WCRT and the
BCRT, respectively that satisfy the real-time wait requirements. The
intersection of these two sets gives all the logical ticks that satisfy
the real-time requirements invariant of the logical time and its
elasticity.

Let us revisit the HRTCS example (cf. Figure~\ref{fig:1}) to illustrate
the algorithm. From Figure~\ref{fig:1} we know that $M$ is 50.3 ms and
$N$ is 200.3 ms, respectively. \hj{Timing analysis of the program has
  found} WCRT and BCRT to be: 0.112 ms and 0.0334 ms,
respectively. Thus, the algorithm proceeds as follows:

\begin{enumerate*}
\item $l_1 \leftarrow \lceil 50.3/0.112 \rceil$ and $u_1 \leftarrow
  \lfloor 200.3/0.112 \rfloor$. $l_1 = 450$ and $u_1 = 1788$. We first
  calculate the logical ticks that are always running at the WCRT and
  satisfy the required real-time wait period.
\item $l_2 \leftarrow \lceil 50.3/0.0334 \rceil$ and $u_2 \leftarrow
  \lfloor 200.3/0.0334 \rfloor$. $l_2 = 1506$ and $u_2 = 5997$. We do
  the same for the BCRT case.
\item $S_1 = [450,1788]$ and $S_2 =[1506,5997]$. We then map the
  resultant bounds to linear points. Sets $S_1$ and $S_2$ represent
  logical ticks that running at the WCRT and BCRT, respectively always
  satisfy the required real-time constraints.
\item $D = S_1 \cap S_2$, $D = [1506,1788]$. Finally, the intersection
  of the two sets gives the set $D$ from which we can choose any value
  for \texttt{d}.
\end{enumerate*}

\begin{figure}[t!]
  \centering
  \includegraphics[scale=0.6]{FF4}
  \caption{The pictorial representation of the solution for the HRTCS system}
  \label{fig:hrtcssoln}
\end{figure}

The resultant value for \texttt{d} gives the number of logical ticks,
which can run at any physical clock-speed, bounded by the BCRT and the
WCRT of the program and still result in the desired real-time wait
period. Figure~\ref{fig:hrtcssoln} gives the solution space for the
aforementioned calculation. The patterned space is the only overlapping
region and hence gives the solution space for the above example. It is
possible that $D$ is an empty set, that means no overlapping region
could be found. We provide solutions for such cases in the next
subsection.

We think this is an elegant solution, because the technique provides
real-time guarantees while preserving the essence (elastic logical tick)
of synchronous and GALS programming prescribed by SystemJ and Esterel
style languages. Moreover, this technique considers non
maximal-parallelism, i.e., the postponement in logical ticks is
calculated after scheduling of CDs or synchronous programs has been
performed. To our knowledge we are the first to do so.

\subsection{Extending the technique to variants of wait}
\label{sec:extend-tehcn-vari}

\paragraph{The \texttt{wait\_atleast(M)} construct}
\label{sec:extend-techn-vari}

The \texttt{wait\_atleast(M)} variant is easily accommodated in the
technique. All one needs to do is find the set $S_1$ and choose a value
from this set.

\paragraph{The \texttt{wait\_exact(M)} construct}
\label{sec:extend-techn-vari}

The \texttt{wait\_exact(M)} variant is more interesting. Like before,
we find sets $S_1$ and $S_2$, and find the intersection of the two sets
to get the value of \texttt{d}. It is possible % (and often likely, as
% suggested by our experiments in
% Section~\ref{sec:experimental-results})
that the resultant set $D$ is empty (also possible in the case of
\texttt{wait\_inbetween(M..N)}, but never possible in case of
\texttt{wait\_atleast(M)}). Note that $S_1$ and $S_2$ can never be empty
sets and will always contain at minimum a single element. For such
cases, we provide two solutions listed below.

% In such a case, we can \textit{relax} the upper bound of the
% \texttt{delay} statement.

\paragraph{Relaxation of the upper real-time bound}
\label{sec:over-appr-relax}

The relaxation technique is shown in Algorithm~\ref{alg:2}.
  
\begin{algorithm}[t!]
  \begin{minipage}{1.0\linewidth}
    \SetAlgoLined
    \KwData{$S_2$, $D$, WCRT}
    \KwResult{d}
    \If {$D = \emptyset$} {
      let $j_{0}$ be the first element of set $S_2$\;
      \ShowLn let $N \leftarrow WCRT \times j_0$\;
      let $d \leftarrow j_0$\;
    }
    \Return d\;
    \caption{Calculating the minimum relaxation of the upper real-time
      bound}
    \label{alg:2}
  \end{minipage}
\end{algorithm}


This algorithm results in the \underline{smallest} relaxation required
for the real-time postponement to be satisfied. The algorithm takes as
input the WCRT and set $S_2$, recall that set $S_2$ represents the
logical ticks required to satisfy the real-time requirement at the
BCRT. We take the very first value from set $S_2$ and multiply it with
the WCRT to get the relaxation $N$. The first element of set $S_2$ is
returned as the logical tick \texttt{d}. The fundamental observation is
that we have to postpone the proceeding statement for at least $M$ units
of real-time, hence, under-approximation is out of question. We can
still over-approximate, but to reduce the resultant error, we should
over-approximate by the least possible value, which is the lower bound
of set $S_2$. Thus, the lower bound is considered to be the only element
shared between the two sets $S_1$ and $S_2$ and accordingly, the upper
real-time bound is relaxed by the multiplication of WCRT and the first
element of set $S_2$.

For example, assume $BCRT=5$ and $WCRT=100$, with a required
postponement of $(200..200)$. Algorithm~\ref{alg:1}, results in $D$
being an empty set, since $S_1=[2]$ and $S_2=[40]$, respectively. In
such a case, the resultant relaxation is: $40 \times 100$ using
Algorithm~\ref{alg:2}, which results in a time of $4000$ units. Such
large over-approximations can be avoided by breaking the critical paths
(worst case ticks) in the program (by inserting \texttt{pause}
constructs), just like in hardware design, which is very
\textit{unsurprising} considering Esterel has its origins in hardware.

\paragraph{Periodic execution of the clock domains}
\label{sec:peri-exec-clock}

The aforementioned relaxation technique maintains the essence of
synchronous and GALS programs -- the logical tick, but in doing so is
unable to provide the exact real-time wait of $M$ time units in case of
\texttt{wait\_exact(M)} statement, or is unable to provide a
postponement in the region of $[M, N]$ time units in the case of
\texttt{wait\_inbetween (M..N)} statement, thereby violating their
semantics. The second option is to execute the CD periodically at the
WCRT. WCRT determines the fastest execution speed of any CD with the
guarantee that no input events from the environment will be missed
(cf. Section~\ref{sec:mapping-logical-time}). Thus, if we simply execute
the CD periodically at the WCRT, we can choose the value of \textit{d}
from set $S1$. This technique allows meeting the synchronous hypothesis,
while at the same time guaranteeing the semantics of the
\texttt{wait\_exact(M)} statement. This technique trades-off reduced
average reaction time against increased power consumption while
providing exact postponement.

We believe that the designer should be allowed to choose from either of
these techniques and hence, provide this option via compiler switches.

\subsection{The tool-chain flow}
\label{sec:tool-chain-flow}

\begin{figure}[t!]
  \centering
  \includegraphics[scale=0.5]{./tool-flow}
  \caption{The tool-chain flow of SystemJ compiler with waits}
  \label{fig:4}
\end{figure}

The interaction of the target execution platform, the tool-chain flow,
and the described high-level technique of introducing real-time waits in
the SystemJ language is shown in Figure~\ref{fig:4}. The numerical
annotations in Figure~\ref{fig:4} show the sequence of steps in the
compilation procedure. We use static analysis techniques
(see~\cite{zli14} for a more in depth look at our techniques) to find
out the WCRT/BCRT of the programs automatically after rewrites and
standard compiler optimization.  It is worth pointing out that since all
rewrites, including the rewrite for the wait constructs, happen before
WCRT/BCRT analysis, this analysis incorporates the affects of the
introduction of the wait statements.  Moreover, since
Algorithms~\ref{alg:1} and~\ref{alg:2} are performed after scheduling
for different constraints such as faster execution time, or reduced
power consumption, we inherently satisfy the criteria of exploiting
non-maximal parallelism. The low-level analysis guarantees a precise
WCRT/BCRT for the logical tick, which is then used for calculating $d$
in Figure~\ref{fig:3}.

Our compiler has a switch that enables over-appro\-ximation as described
in Section~\ref{sec:extend-tehcn-vari} provided the wait requirements
are not met. If this compiler switch is not enabled, the compiler simply
decides to execute the CD periodically, where the period is the
WCRT. WCRT effects the over-approximation and the period of the CD,
irrespective of the solution that the system designer chooses. A large
WCRT value results in reduced performance and hence it is essential that
the WCRT be reduced. One can reduce the WCRT value of any CD by breaking
the critical paths in the program using the \texttt{pause} construct as
we show later in Section~\ref{sec:experimental-results}.

In the following subsections we describe: (1) the time analyzable target
architecture and the corresponding SystemJ program execution model and
(2) the static analysis techniques used to compute the WCRT/BCRT of a
SystemJ program.

\subsubsection{Time analyzable architecture and the SystemJ program
  execution model}
\label{sec:time-analyz-arch}

SystemJ programs integrate both control-driven and data-driven
operations. Implementing both these types of operations on a single
processor has been shown to be
inefficient~\cite{DBLP:journals/tecs/SalcicM13}, generally resulting in
large code size and longer execution times. In order to achieve
efficient execution, the control-driven and data-driven operations are
separated at the compiler back-end and then mapped onto dedicated
processing units; a control processor and a data processor,
respectively~\cite{DBLP:journals/tecs/SalcicM13}. Both processors work
in tandem, hence the resultant architecture is also termed the
\textit{Tandem Processor} (TP).

The control flow of individual reactions within each CD and the
concurrent execution of multiple reactions within a CD is termed
Concurrency and Reactivity Control Flow (CRCF). Besides control flow
within the reactions, CRCF also includes scheduling of all reactions
within the CDs, communication between reactions in different CDs, and
communication between reactions and their respective environments. CRCF
code is executed on the control processor. On the other hand, the
control flow of data-driven operations is called Java Control Flow
(JCF), since all data computations of SystemJ programs are compiled to
Java bytecodes running on the data processor. With this separation,
SystemJ program execution is initiated and directed by the CRCF;
whenever a Java data computation is required, the execution is switched
to JCF and returns to CRCF again upon completion.

\begin{figure}[t!]
  \centering
  \includegraphics[scale=0.8]{ArchG}
  \caption{The target TP-JOP execution architecture}
  \label{fig:8}
\end{figure}

In this work we choose a multi-core platform named TP-JOP
(cf. Figure~\ref{fig:8}) as our target execution platform. TP-JOP
execution platform uses Reactivity and COncurrency Processor
(ReCOP)~\cite{DBLP:journals/tecs/SalcicM13} as the control processor and
Java Optimized Processor (JOP)~\cite{jop:jnl:jsa2007} as the data
processor. TP-JOP is based on the idea of Tandem Processor (TP) proposed
in~\cite{DBLP:journals/tecs/SalcicM13} with different implementation for
the data processor. Both ReCOP and JOP are designed with time-analyzable
features, which make WCRT/BCRT analysis of SystemJ programs feasible for
TP-JOP. % The block diagram of TP-JOP is shown in Figure~\ref{fig:4} --
% denoted as \textit{Architecture Graph}.  Note here in our work there
% is no interleaving between the execution of ReCOP and JOP, resulting
% in a sequential execution pattern that simplifies the timing analysis.

JOP is a real-time analyzable soft-core processor targeted at execution
of Java programs in an embedded environment~\cite{jop:jnl:jsa2007}. A
\textit{Worst Case Execution Time} (WCET) analysis tool called JOP WCET
Analyzer (WCA) is provided with JOP. WCA performs static WCET analysis
of Java program based on Implicit Path Enumeration (IPET) approach
[15]. In our framework, the WCA tool is used to analyze the WCET of all
data-driven operations in the JCF. ReCOP is designed exclusively to deal
with the execution of CRCF of SystemJ programs. It also handles the
communication with JOP and with the environment. ReCOP executes the CRCF
assembly code generated by the back-end of the SystemJ compiler. Every
assembly instruction, generated from CRCF, is executed in exactly three
ReCOP clock cycles; fetch, decode and execution cycle.

\subsubsection{Static analysis for computing the WCRT and BCRT of a
  SystemJ program}
\label{sec:stat-analys-comp-1}

\begin{figure*}[t!]
  \centering
  \begin{SubFloat}{\label{fig:6a} TAGRC of the HRTCS SystemJ program}
    \includegraphics[scale=0.44]{AGRC}
  \end{SubFloat}%
  \begin{SubFloat}{\label{fig:6b} TA of first CD in HRTCS}
    \includegraphics[scale=0.45]{Uppaal}
  \end{SubFloat}
  \caption{TAGRC of the HRTCS program and its corresponding TA
    translation}
  \label{fig:6}
\end{figure*}

Before giving a detailed description of the static analysis technique
used to calculate the WCRT/BCRT of a SystemJ program we revisit the
tool-chain flow, in Figure~\ref{fig:4}, to give an overview of the
compilation and WCRT/BCRT estimation procedure.

In step 1, the source code of a given SystemJ program is first compiled
into Asynchronous GRaph Code (AGRC) intermediate representation using
the front-end of the SystemJ compiler~\cite{amal10} (see
Figure~\ref{fig:6a} for an example of the AGRC intermediate format for
the HRTCS running example). The AGRC representation at this step does
not contain any timing information and is then sent to the back-end of
the SystemJ compiler for target code generation. The target code
generated by SystemJ compiler for TP-JOP consist of assembly
instructions, representing CRCF code, executed on ReCOP and bytecodes,
representing JCF code, executed on JOP (step 2). During target code
generation in step 2, the tracking information is also produced,
indicating the original AGRC node from which a segment of target code is
generated.

Both CRCF assembly code and JCF bytecode are analyzed by our customized
WCET analyzer to calculate WCET information for each AGRC node (step
3). The customized WCET Analyzer used in step 3 comprises of the
standard JOP WCA~\cite{jop:jnl:jsa2007} and ReCOP WCET analyzers. The
latter performs WCET analysis for CRCF code by counting the number of
assembly instructions for any AGRC node and deriving the WCET result
based on the micro-architecture level analysis of ReCOP. The original
AGRC is then back-annotated with the node-level WCET information
acquired from step 3 to obtain Timed Asynchronous GRaph Code (TAGRC). 


One needs the WCET value of each node in the AGRC for computing the
overall WCRT for each CD. Similarly, the \textit{Best Case Execution
  Time} (BCET) of each AGRC node is needed when computing the BCRT of
each CD. If these nodes are basic-blocks -- sequential code with single
entry and exit points without branches, the WCET and BCET of the node
are same. If the node itself contains branches then WCET and BCET of
each node is different and needs to be computed as such. The compiler
can make sure that each AGRC node is indeed a basic-block (so that the
WCA analysis tool for JOP and the WCET analyzer for ReCOP remain
unchanged) and all (CRCF and JCF) branches are represented as explicit
edges in the AGRC.

AGRC extends the GRaph Code (GRC)~\cite{pbat07} with asynchrony by
composing separate GRCs, one for every CD in the program. In order to
compartmentalize our WCRT/BCRT analysis to each synchronous CD, TAGRC is
partitioned into a set of Timed GRaph Codes (TGRCs), one for every
CD. After obtaining TGRCs, model checking based tight WCRT/BCRT analysis
is carried out in steps 4 and 5. In these steps, each TGRC obtained from
step 3 is translated into a Timed Automaton (TA) (see
Figure~\ref{fig:6b} for the TA generated from the HRTCS example) and
input into the Uppaal~\cite{gbeh04} model-checker for exhaustive path
exploration to find a tight WCRT/BCRT estimate of the program. Instead
of using real-valued clocks, we employ a single bounded integer to
capture the total time cost of a complete transition from the beginning
to the end of a logic tick.

During the translation from TGRC to TA we preserve the following
information: (1) the state encoding and decoding that capture tick
transition semantics and (2) internal and output signal emission and
testing. Model checking based WCRT/BCRT analysis benefits from
preserving this information, which essentially models the execution flow
of the SystemJ program, because model-checker can implicitly prune paths
via control flow analysis. Consequently a tight WCRT/BCRT result can be
obtained. % We exploit a well-known model checker for timed automata
% called UPPAAL~\cite{gbeh04}, because it offers efficient algorithms for
% both TA and automata with integer operations. In our static analysis
% framework, shown in Figure~\ref{fig:4}, all the timed automata are
% generated in the format of UPPAAL model.

\subsubsection{Model-checking based tight WCRT and BCRT analysis}
\label{sec:model-checking-based}

Figure~\ref{fig:6a} shows the TAGRC for our running example: the HRTCS
system~\footnote{The TAGRC for HRTCS has been abstracted for the sake of
  understanding.}. TAGRC is so called, because every node in the AGRC is
annotated with WCET information, obtained via low level
micro-architectural analysis. AGRC is the intermediate presentation that
captures the semantics of the original SystemJ program. It is obtained
by applying \textit{Structural Operational Semantic} rules to the
original SystemJ program~\cite{amal10}.

AGRC consists of a number of unique nodes: the \textit{Afork Node} forks
multiple CDs, the \textit{Fork Node} is responsible for forking
synchronous parallel reactions, the forked synchronous parallel
reactions are synchronized together by the \textit{Join Node}. The CDs
on the other hand being asynchronous to each other do not block at the
\textit{Ajoin Node}, instead the \textit{Ajoin Node} is used to perform
input output communication between the CD and its environment via
signals. The current state of every reaction within any CD is encoded
and later decoded via the \textit{Enter Node} and \textit{Switch Node},
respectively. Similarly, the current termination context of the
reactions is captured via \textit{Termination Node}. Finally,
instantaneous actions such as emit, and Java data-computations are
represented as \textit{Action Node} and branching constructs, for
testing status of signals and result of data computation, are
represented with the \textit{Test Node}. A single execution of the AGRC
from the \textit{AforkNode} to the \textit{Ajoin Node} represent a
single tick transition for any given CD in the original SystemJ program.
Different branches of \textit{Switch Node} and \textit{Test Node} might
be traversed, for any given tick transition, depending upon the current
state of the program including signal statuses and values. On the other
hand, every branch of a \textit{Fork Node} is executed in every tick
transition. % Moreover, the order of execution of these \textit{Fork
            % Node}
% branches is irrelevant (see~\cite{amal10} for more details).

Figure~\ref{fig:6b} shows the TA~\footnote{Timed Automaton} of CD 1 from
Figure~\ref{fig:6a}. This TA is input into the Uppaal model-checker for
finding out the WCRT/BCRT of each CD. The first step in this translation
is to perform a one-to-one mapping for each node in the TGRC to a
location in the TA. For example, \textit{Afork Node} AF0 in the TGRC is
mapped to the initial location AF0 in the TA. Similarly, \textit{Switch
  Node} S1 is mapped to a location S1, and so on and so forth. Next, we
map each control flow edge in the TGRC to a transition between two
locations in the TA. The conditional branching is modeled by annotating
the transitions with additional guards. For instance, the transition
from location S1 to E1 in Figure~\ref{fig:6b} has the expression
\texttt{S1==0} as its guard condition to capture the tick transition
semantics based on state encoding and decoding of switch node S1 in
Figure~\ref{fig:6b}. The signal emission and state encoding process is
captured by the assignments on the transition in the TA. In
Figure~\ref{fig:6b}, assignments on transitions from E1, E2 and E3, etc
model three distinct state encodings.

As stated previously, a bounded integer (\texttt{wcrt}) is used to count
the total execution time for a single tick transition for any given
CD. Therefore, \textit{every} transition in the TA is annotated with an
assignment that increases the integer \texttt{wcrt} by the WCET value of
the target location. Note that WCRT value of a target node might be
zero. For example the \textit{Join Node} has a WCET value of zero. Due
to the semantics of AGRC nodes related with control flow, modifications
are needed to the TA to correctly model the execution flow. We use
cyclic executive scheduling (see~\cite{amal10} for further details) for
synchronous parallel reactions within a CD. This cyclic execution flow
of synchronous parallel reactions is captured in the TA by inserting a
back-edge from the \textit{Join Node} (e.g., J2) to its corresponding
\textit{Fork Node} (e.g., F2). Furthermore, since a tick transition for
every CD starts at \textit{Afork Node} and finishes at the \textit{Ajoin
  Node}, there should be a transition from \textit{Ajoin Node} back to
\textit{Afork Node} to model the repeating CD tick transitions. The
bounded integer \texttt{wcrt} also needs to be cleared during this
transition. Finally, communication of this CD with its environment is
also modeled as signal status and value updates on this transition.

We model the tight WCRT/BCRT analysis problem as verifying a
\textit{Computational Tree Logic} (CTL) property in UPPAAL model
checker. Herein we explain how the tight WCRT is obtained. Obtaining the
tight BCRT follows an analogous procedure. The tight WCRT, named
WCRTtight, is the objective of WCRT analysis. WCRTtight lies in the
bounded range denoted by [WCRTlb, WCRTub]. WCRTub is a safe upper-bound
of WCRT obtained by applying Max-Plus algebra~\cite{boldt07} to the
TGRC, which involves essentially summing up the maximum tick execution
time of every reaction in the CD. Similarly, we calculate a lower-bound
of WCRT, termed WCRTlb, by summing up the minimum tick execution time of
every reaction in the CD. After translating TGRC to the TA, we check the
validity of a WCRT estimate of the CD, termed WCRTest, by verifying a
CTL property upon the TA using UPPAAL model checker. The CTL property is
written as \texttt{A[](wcrt $\leq$ WCRTest)}, meaning that the value of
integer \texttt{wcrt} is less or equal to some value WCRTest for every
path starting from the initial location. In order to minimize the number
of queries, we use standard binary search algorithm to find
WCRTtight. For the very first query in this binary search procedure,
WCRTest is set to $\frac{WCRTub - WCRTlb}{2}$. From the second query on
wards, WCRTest is recalculated depending upon the result of the previous
query: satisfied or not satisfied.


% The system designer can then either set the compiler switch or change
% the program by shortening the critical path by introducing
% \texttt{pause} constructs. We show how introducing \texttt{pause}
% constructs can help reduce the WCRT in
% Section~\ref{sec:experimental-results}.


%%% Local Variables: 
%%% mode: latex
%%% TeX-master: "template"
%%% End: 
