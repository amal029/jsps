\section{Related work}
\label{sec:related-work}


A number of studies have been performed when it comes to integrating
real-time directly as a programming construct in GALS (and its subset
synchronous) reactive languages. The most prominent works in this area
are by Shyamsundar~\cite{rsh94} and Bourke et al.~\cite{Bourke2009a}.
Shyamsundar incorporates real-time using external timers in
\textit{Communicating Reactive Processes} (CRP), which like SystemJ, is
an extension of synchronous language Esterel to asynchorny. But, as
mentioned in Section~\ref{sec:inter-timers-react}
and~\ref{sec:resolution-real-time} external timers do not interact well
with preemption constructs in these languages. Bourke et al. introduce
real-time as first class constructs in the Esterel language. They like
us provide real-time wait statements called \texttt{delay} as first
class Esterel programming constructs, and translate them into Esterel
kernel constructs. But, unlike us they do not translate wait statements
into \texttt{pause} constructs directly. Instead, logical ticks are
generated by using abstract notion of \texttt{event} and \texttt{sample}
platform dependent timers. This notion makes it a complex and inflexible
solution since the number of platform timers with certain resolutions
need to be determined and present on the system for the solution to be
realizable. Moreover, non exact real-time waits and integration with
non-maximal parallelism is not studied at all. Our solution does not
require external timers \cite{rsh94} or logical notion of timers
\cite{Bourke2009a}. In our solution we calculate the real waiting time
in logical ticks \textit{after} allocation and scheduling, it gives the
compiler developers the chance to optimize for many different criteria
such as computation time, energy, power, etc, without worrying about or
violating the real-time wait specification. All the aforementioned
approaches \cite{rsh94,Bourke2009a} as well as our technique presented
in this paper cha\-nge the internal automata that capture the semantics
of the program. This is as expected as programmers introduce additional
constructs in order to incorporate the real-time postponements in the
program. However, as explained in Section~\ref{sec:disc-perc-limit}, our
approach does not face the problems such as deadlock due to interaction
with external timers.


Bertin et al.~\cite{Bertin:2000:TVR:1947412.1947439} also incorporates
real-time wait statements in Esterel using special \textit{pragmas}
(annotations). They translate real-time pragmas into timed
automata~\cite{alur94} for model-checking real-time deadlines. Our
solution on the other hand removes all continuous time elements and
produces single discrete time model thereby making the solution amenable
to existing functional and real-time verification tools developed for
reactive languages and more in the spirit of logical time. Moreover, a
major difference between Bertin et al. and our approach is that their
approach requires programmers to annotate the time taken by `C' code
fragments manually, this is extremely error prone, we use static
low-level program analysis to find out such times automatically.
Lastly, Logothethis et al.~\cite{glog02} uses the \texttt{pause}
statement in a variant of the Esterel language called Quartz.
Nevertheless, their solution is targeted at studying timing properties
using model checkers rather than implementation, like us.













%%% Local Variables: 
%%% mode: latex
%%% TeX-master: "paper"
%%% End: 


